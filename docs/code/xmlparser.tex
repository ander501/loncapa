\chapter{XML Parser}\index{XML parser}\label{xmlparser}
\section{Invocation}
The XML parser is at the heart of much of LON-CAPA's operation. As opposed to the original LON-CAPA, it does not distinguish between problems and non-problems, and it also serves system pages.\index{problems}\index{HTML pages} It is called via lc.conf:
\begin{verbatim}
<LocationMatch "\.(xml|html|htm|xhtml|xhtm)$">
SetHandler perl-script
PerlHandler Apache::lc_asset_xml
</LocationMatch>
\end{verbatim}
The main handler is thus lc\_asset\_xml. All other modules need to register their tags with this parser.

The XML parser actually since the early days of LON-CAPA used an HTML parser,
\begin{verbatim}
use HTML::TokeParser();
\end{verbatim}
so it is more robust against trying to make sense of malformed XML, however, no malformed document should be stored. As this is the same parser mechanism as
is used in LON-CAPA 2.x, porting of the extensive homework functionality will be facilitated. Documentation for this module is online.
\section{Tag types}
Tags are defined via subroutines. The name of these subroutines has three parts: start or end, tagname, and target. For example, here is lc\_xml\_localize, which defines the localize-tag:
\begin{verbatim}
package Apache::lc_xml_localize;

use strict;
use Apache::lc_ui_localize;

our @ISA = qw(Exporter);

# Export all tags that this module defines in the list below
our @EXPORT = qw(start_localize_html start_localize_tex);

sub start_localize_html {
   my ($p,$safe,$stack,$token)=@_;
   my $text=$p->get_text('/localize');
   $p->get_token;
   pop(@{$stack->{'tags'}});
   return &mt($text,
              split(/\s*\,\s*/,&Apache::lc_asset_safeeval::texteval($safe,$token->[2]->{'parameters'})));
}

sub start_localize_tex {
   my ($p,$safe,$stack,$token)=@_;
   return &mt($p->get_text('/localize'),
              split(/\s*\,\s*/,&Apache::lc_asset_safeeval::texteval($safe,$token->[2]->{'parameters'})));
}

1;
__END__
\end{verbatim}
Target html is for web, target tex for printing. Defined tags need to be exported. In lc\_asset\_xml, lc\_xml\_localize needs to be used (without parenthesis).

If a tag is not defined, it is just printed to the screen on the web and ignored in print.

\section{Stacks and safeeval}\index{safeeval}
The stack is used to store the arguments of tags, so they can be retrieved later or inside of nested tags. The safeeval-space is a Perl safespace in which calculations and storage can happen (same as in the old LON-CAPA).


