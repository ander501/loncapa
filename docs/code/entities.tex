\chapter{Entities}
\section{Identification}\index{entities}\label{entities}
Users, courses, and assets are entities in the system. Since the network has domains, an entity is identified by two components:
\begin{itemize}
\item An entity code (in the system internally often referred to as just ``entity'', e.g., ``qLLTNbdEhQaxQ8AZyYp''
\item The domain (e.g., ``msu'')
\end{itemize}
Thus, entities should {\it always} be refered to by both, qLLTNbdEhQaxQ8AZyYp:msu. Just specifying the entity code is not enough!

Entity codes are guaranteed to be unique within a domain, but not between domains.

Entities do not change. An individual, course, or assets always keeps its entity:domain.
\section{Storage and Interface}
Information in the system is always stored by entity:domain, not by usernames or course codes. However, entity:domain is not exposed on the interface level, here we talk in terms of usernames, personal ID numbers, or course codes.
\section{Users}\index{users}\index{usernames}\index{PIDs}
Usernames and personal ID numbers (PIDs) are properties of the user entity. They need to be unique within a domain, thus to fully specify a user entity in this fashion, one needs to specify the username {\it and} domain. Just specifying the username is not enought!

Usernames and PIDs can change. For example, username changes can happen due to live events or because the user wants a vanity ID. At most universities, usernames once given out are never recycled. Also, PIDs can change if a person changes from student to faculty/staff or vice versa. Once again, PIDs at most places are never recycled.

The system should thus assume:
\begin{itemize}
\item At any given point in time, users have one primary username and PID.
\item The system needs to still support the old usernames and PIDs, as particularly during mid-semester changes, uploadable grading lists or other spreadsheets may still include the old username/PID
\end{itemize}

Each username or PID points to one and only one user entity. However, a user entity can have more than one supported username or PID.
\section{Courses}\index{courses}
Courses are treated very similarily to users. They have course IDs such as ``phy231fs14'', which might have. These are treated the same way as usernames or PIDs.

Communities are special kinds of courses which do not have grade book, and which have different associated roles. Internally, they are distinguish by the ``type'' in the course profile being ``community'' instead of ``regular.''\index{communities}

Courses can have sections and groups.\index{sections}\index{groups}
\begin{itemize}
\item Sections usually correspond to educational venues such as laboratories or recitations. A student can only be in one section at a time.
\item Groups are more like teams, working together. A student can be in more than one group at a time
\end{itemize}
Internally the two are distinguished by ``type'' in the course/community profile.
\section{Assets}\index{assets}\label{assets}
\subsection{URLs}
Assets are HTML pages, problems, images, movies, etc, which are stored in user portfolios.\index{portfolio}
Instead of usernames, PIDs, or couse IDs, assets have URLs. One asset can have more than one URL, however, the URL of published assets looks like
\begin{center}
{\tt /asset/n/3/msu/smith/...}
\end{center}
where the first two components after ``asset'' designate the version, followed by the domain and the publishing author.

The above example explictly asks for version number 3 of the asset. Other ways are
\begin{center}
{\tt /asset/as\_of/2014-01-08\_04:05:06/msu/smith/...}
\end{center}
which asks for the version as-of a certain date. Finally,
\begin{center}
{\tt /asset/-/-/msu/smith/...}
\end{center}
asks for the most recent version.\index{versions}

If at all possible, relative URLs should be used, so version arguments like ``as\_of'' get preserved to also get the right versions of dependent files.
\subsection{Profiles and metadata}\index{profiles}\index{metadata}
Both users and courses have profiles, which store basic information like their full name or title, or any configuration preferences. Assets have metadata, whcih is used for various cataloging purposes and populated with both static and dynamic metadata. These are stored in MongoDB.\index{MongoDB}.
\subsection{Storage}
Assets are physically stored on the file system under their entity, not their URL,  see Section~\ref{resfilesystem} on page~\pageref{resfilesystem}. An asset entity can have more than one URL. The URLs are a property of the entities and translated using the translation handler, see Section~\ref{accassets} on page~\pageref{accassets}.

The authoritative copy of assets sit on the author's homeserver.\index{homeserver} During replication (Section~\ref{replication} on page~\pageref{replication}), copies are made between servers.
\subsection{Publication and Versioning}\index{publication}\index{versions}
Assets move from the ``\_wrk'' version to the real next version during the process of publication.\index{publication}
 As a new version of a resource is generated, this triggers the updating across servers through replication (see Section~\ref{replication} on page~\pageref{replication}).
