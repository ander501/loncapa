\chapter{URL Translation}
\section{Accessing handlers}
Handlers are defined in lc.conf using Location and LocationMatch statements.\index{lc.conf} For example
\begin{verbatim}
<Location /menu>
SetHandler perl-script
PerlAccessHandler Apache::lc_auth_optional
PerlHandler Apache::lc_ui_menu
</Location>
\end{verbatim}
defines how the URL /menu is handled. During the Apache request cycle, first the PerlAccessHandler is called. There are two versions,
\begin{itemize}
\item lc\_auth\_acc, which requires a session and loads the session environment - otherwise, an error handler usually points to the login handler
\item lc\_auth\_optional, which only loads a session environment if a session is active
\end{itemize}
The statement then defines lc\_ui\_menu as the response handler. Additional error handlers can be called, for example to route back to the login screen.

Some handlers also have cleanup handlers, which may do the actual work, for example
\begin{verbatim}
<Location /async>
SetHandler perl-script
PerlAccessHandler Apache::lc_auth_acc
PerlHandler Apache::lc_ui_async
PerlCleanupHandler Apache::lc_ui_async::main_actions
</Location>
\end{verbatim}
This handler can reply to the browser immediately and then do the actual work in the cleanup phase. This is most useful in connection with asynchronous AJAX requests.
\section{Accessing assets}\index{assets}\label{accassets}
Access to assets is more complicated. They are caught in the global translation handler
\begin{verbatim}
PerlTransHandler Apache::lc_trans
\end{verbatim}
which (at some point) looks/looked like this:
\begin{verbatim}
sub handler {
    my $r = shift;
# We care about assets
    if ($r->uri=~/^\/asset\//) {
# First check if we can even find this
       my $filepath=&Apache::lc_entity_urls::url_to_filepath($r->uri);
       unless ($filepath) {
# Nope, this does not exist anywhere
          return HTTP_NOT_FOUND;
       }
# Is this locally present?
       unless (-e $filepath) {
# Nope, we don't have it yet, let's try to get it
          unless (&Apache::lc_entity_urls::replicate($r->uri)) {
# Wow, something went wrong, not sure why we can't get it
             &logwarning("Failed to replicate ".$r->uri);
             return HTTP_SERVICE_UNAVAILABLE; 
          }
       }
# Bend the filepath to point to the asset entity
       $r->filename($filepath);
       return OK;
    } elsif ($r->uri=~/^\/raw\//) {
       $r->filename(&Apache::lc_entity_urls::raw_to_filepath($r->uri));
       return OK;
    } 
# None of our business, no need to translate URL
    return DECLINED; 
}
\end{verbatim}
This deals with URLs of the type /asset and /raw, and does the translation of URL-paths to the asset entity:domain, as well as versioning and replication. By the time this handler is done, the asset is either present or we throw a NOT\_FOUND.\index{replication}\index{versions}
\section{System Pages}
System pages such as the dashboard or the preferences are actually HTML-pages, which include LON-CAPA XML. They use the same XML parser\index{XML parser} as for example homework pages. An example is lc\_preferences.html, the user preferences screen:
\begin{verbatim}
<html>
<head>
<title>Preferences</title>
<script src="/scripts/lc_preferences.js"></script>
</head>
<body>
<h1><localize>Preferences</localize></h1>
<p>
<span class="lcstandard"><localize>You can modify your user preferences below.</localize></span>
<span class="lcerror"><localize>A problem occured, please try again later.</localize></span>
<span class="lcproblem"><localize>A problem occurred while saving your preferences.</localize></span>
<span class="lcsuccess"><localize>Your preferences were saved.</localize></span>
&nbsp;
</p>
<lcform id="preferencesform">
<lcformtable>
<lcformtableinput type="language" description="Language" id="language" default="user" />
<lcformtableinput type="timezone" description="Timezone" id="timezone" default="user" />
</lcformtable>
<lcformtrigger id="storebutton" description="Store" />
</lcform>
</body>
</html>
\end{verbatim}
\section{Deep Links}\index{deep linking}
Deep linking works via the {\tt lc\_ui\_direct} handler. It utilizes the token-mechanism (see Section~\ref{token} on page~\pageref{token}).

The URLs for deep linking start with {\tt /direct}.
