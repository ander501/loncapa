\chapter{Storage}
\section{Databases}
The system uses two databases: PostgreSQL and MongoDB.\index{PostgreSQL}\index{MongoDB}
\begin{itemize}
\item PostgreSQL is a traditional relational database which is used for predictable data that can reside in tables. These are often lookup-tables that need fast search
\item MongoDB is a noSQL database which is used for flexible, structured data. Searches are rare and not performance-critical.
\end{itemize}
Data only sits on the homeserver of the user or course.\index{homeserver}\index{users}\index{courses}
\subsection{PostgreSQL tables}
The tables in PostgreSQL are
\subsubsection{URL table}
The table is used to look up the entity code for a certain URL. More than one URL can point to the same asset. The domain is not stored, since it is already encoded in the URL.
\begin{verbatim}
#
# Make the URLS table
#
create table urls
(url text primary key not null,
entity text not null)
\end{verbatim}

\subsubsection{User table}
This table is used to look up the entity code of usernames.\index{usernames} More than one username can point to the same entity code. Both entity code and username would be in the same domain.
\begin{verbatim}
#
# Make the user lookup table
# Get the entity for a username
#
create table userlookup
(username text not null,
domain text not null,
entity text not null,
primary key (username,domain))
\end{verbatim}

\subsubsection{PID table}
Same as user table for PIDs.\index{PIDs}
\begin{verbatim}
#
# Make the pid lookup table
# Get the entity for a PID
#
create table pidlookup
(pid text not null,
domain text not null,
entity text not null,
primary key (pid,domain))
\end{verbatim}
\subsubsection{Course ID table}
Same as username and PID tables for course IDs.\index{course IDs}
\begin{verbatim}
#
# Make the courseID lookup table
# Get the entity for a courseID
#
create table courselookup
(courseid text not null,
domain text not null,
entity text not null,
primary key (courseid,domain))
\end{verbatim}
\subsubsection{Homeserver table}
Table used to look up the homeserver\index{homeserver} of an entity:domain.
\begin{verbatim}
#
# Make the homeserver lookup table
#
create table homeserverlookup
(entity text not null,
domain text not null,
homeserver text not null,
primary key (entity,domain))
\end{verbatim}
\subsubsection{Roles}\index{roles}\label{rolelookup}
This is for looking up who has certain roles. It is not the authoritative version, the authoritative record is stored with the user (see Section~\ref{rolerecord} on page~\pageref{rolerecord}).
\begin{verbatim}
#
# The role table
# These are the roles on this server
# The primary cluster server (and only the primary cluster server)
# also has the system and domain-wide roles
# This is for lookup, the actual roles are with the users
#
create table rolelist
(roleentity text,
roledomain text,
rolesection text,
userentity text not null,
userdomain text not null,
role text not null,
startdate timestamp,
enddate timestamp,
manualenrollentity text,
manualenrolldomain text,
primary key (roleentity,roledomain,rolesection,userentity,userdomain,role))
\end{verbatim}
\subsubsection{Assessment table}\index{assessments}
This has the standardized data about assessments in courses, which is used for gradebooks. More structured data to be called up when the history of a problem is desired, or if it is to be brought up on the screen, resides in MongoDB.
\begin{verbatim}
#
# This is the big table of course assessments on the homeserver of the courses
# Authoritative
#
create table assessments
(courseentity text not null,
coursedomain text not null,
userentity text not null,
userdomain text not null,
resourceid text not null,
partid text not null,
scoretype text,
score text,
totaltries text,
countedtries text,
status text,
responsedetailsjson text,
primary key (courseentity,coursedomain,userentity,userdomain,resourceid,partid))
\end{verbatim}
\section{Caching}\index{memcached}\index{caching}
The system uses two caching mechanisms: in-memory and Memcached
\begin{itemize}
\item In-memory is occasionally used if a particular process needs to preserve variables, so they do not need to be reinitialized every time. Examples are database handles, etc.
\item Memcached is the main caching mechanism, which caches data in memory across processes. The cache items have associated expiration times. The time to refresh these caches is distributed across processes and users.
\end{itemize}
\section{File system}
Assets are stored on the file system in /home/loncapa/res. The authoritative copy of assets is on the author's homeserver (see section~\ref{homeserver} on page~\pageref{homeserver}), but they are copied through replication (see section~\ref{replication} on page~\pageref{replication}) to other servers.\index{homeserver}\index{replication}

The filesystem uses the entity code as part of the location. For example,
\begin{verbatim}
/home/loncapa/res/msu/M/4/O/W/M4OWjvPTQCIE6o8RTMJ_3
\end{verbatim}
is version 3\index{versions} of the asset with entity code M4OWjvPTQCIE6o8RTMJ in domain msu.\index{assets}

\section{Programmatic access}
Databases should not be touched by any higher order handlers. While drivers are located in the /databases GIT directory, e.g., lc\_memcached.pm  lc\_mongodb.pm  lc\_postgresql.pm, these should not be called. Instead, handlers should access the abstractions in the /entities GIT directory, and there only the routines that are not starting with ``remote'' or ``local''.
