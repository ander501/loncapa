\lstset{keywordstyle=\color{blue}}
\lstdefinelanguage{apache}{morekeywords={Location, LocationMatch}}

\chapter{Interface Programming}
\label{interfaceprogramming}
\section{Interface Pages}


\subsection{General Mechanism}
Interface pages (as opposed to asset pages) in LON-CAPA are typically controlled by an HTML, a JavaScript, and a Perl script file located in the {\tt loncapa/app} directory.
\begin{itemize}
\item  {\tt loncapa/app/html/pages} and {\tt loncapa/app/html/modals} contain the HTML/XML framework which is displayed to the user. The page contains the general layout, and it combines HTML-tags with LON-CAPA XML-tags. As this
page is processed by the parser (see Chapter~\ref{xmlparser} on page~\pageref{xmlparser}), the LON-CAPA tags are replaced by dynamically generated HTML on the way to the browser.
\item The HTML file will usually load JavaScript files located in {\tt loncapa/app/scripts} to control client-side functionality.
\item As the scripts make AJAX requests\index{AJAX} to update parts of the page, server-side functionality is provided by Perl modules located in {\tt loncapa/app/handlers}. These modules often return simple text or JSON,\index{JSON} which is processed client-side and injected into the page.
\end{itemize}
In rare cases, XML-tags establish areas which are not updated by client-side JavaScript, but which require server-side functionality. Usually, this is the case when server-provided data is not
structured enough to map well into JSON, e.g., when targeted follow-up questions in various formats need to be generated on the fly. The handlers for this server-side generated HTML, which is then
directly injected into the page, are located in {\tt loncapa/xml/xml\_includes}.


\subsection{Examples}
Let's walk through a couple examples to see how these different files are used.  We begin at the homepage of the site which loads {\tt index.html}.  This sets the general structure of the page and defines space for things like the page header and the menu.  It also loads the JavaScript file, {\tt lc\_default.js}, which includes common JavaScript functions.  One such function that is called upon loading is {\tt menubar()}.  This function dynamically generates the menu based on the context.  It communicates with the server through a JSON call, {\tt \$.getJSON("menu",...)}.  {\tt  lc.conf} directs all "menu" requests to the Perl handler, {\tt lc\_ui\_menu.pm}.  The handler returns a JSON object containing the appropriate menu data.

Another function in {\tt lc\_defaults.js} is {\tt display\_asset(newuri)} which replaces the main iframe with the contents of the passed uri.  {\tt lc\_default.js} also calls the JavaScript function, {\tt dashboard()}, which loads the Dashboard.  It uses {\tt display\_asset()} to load {\tt lc\_dashboard.html} into the iframe.

\subsubsection{Portfolio}
Now let's see what happens when we go to the Portfolio space.  Clicking on the menu option, Portfolio, calls the JavaScript function, portfolio(), which uses {\tt display\_asset()} to load {\tt lc\_portfolio.html} into the iframe.  {\tt lc\_portfolio.html} loads {\tt lc\_portfolio.js}, containing additional JavaScript functions needed in the portfolio space.  {\tt lc\_portfolio.html} also creates the datatable with the following code: 

\begin{lstlisting}[language=HTML,frame=single,title=lc\_portfolio.html]
<lcform id="portfolio" screendefaults="portfolio">
  <lcformtable id="paths">
    <span class="lcerror"><localize>A problem occured, please try again later.
        </localize></span>
    <lcformtableinput description="Path" id="pathrow" type="portfoliopath" />
  </lcformtable>
  <lcfileupload id="newfile" target='/upload_file' successcall="uploadsuccess" 
      failurecall="uploadfailure" />
  <lcdatatable id="portfoliolist" class="portfoliomanager" pathfield="path" />
</lcform> 
\end{lstlisting}

Upon loading, {\tt lc\_portfolio.js} calls {\tt init\_datatable()} which populates the user's portfolio.  It does so with the following command:

\begin{lstlisting}[language=Java,frame=single,title=lc\_portfolio.js]
$('#portfoliolist').dataTable( {"sAjaxSource" : '/portfolio?'+$('#portfolio').serialize()+
    '&command=listdirectory&showhidden='+showhidden+'&noCache='+noCache,...  
\end{lstlisting}

\begin{sloppypar}
The beginning of this statement, {\tt \$('\#portfoliolist').dataTable(}, selects the table with id ``portfoliolist'' defined in {\tt lc\_portfolio.html} and populates it with the results of the {\tt dataTable()} call.  The first argument of {\tt dataTable()}, {\tt sAjaxSource}, contains the URL from which the data is pulled.  {\tt lc.conf} directs this request to the handlers {\tt lc\_auth\_acc.pm} and {\tt lc\_ui\_portfolio.pm}.
\end{sloppypar}

\begin{lstlisting}[language=apache,frame=single,title=lc.conf]
<Location /portfolio>
SetHandler perl-script
PerlAccessHandler Apache::lc_auth_acc
PerlHandler Apache::lc_ui_portfolio
</Location>
\end{lstlisting}

{\tt lc\_auth\_acc.pm} passes the query string from {\tt sAjaxSource} along to {\tt lc\_entity\_sessions.pm}.

\begin{lstlisting}[language=Perl,frame=single,title=lc\_auth\_acc.pm]
sub handler {
  my $r = shift;
  return &get_session($r);
}

sub get_session {
  my $r = shift;
  ...
  &Apache::lc_entity_sessions::get_posted_content($r);
  ...
}
\end{lstlisting}

\begin{lstlisting}[language=Perl,frame=single,title=lc\_entity\_sessions.pm]
...
use vars qw($lc_session);

sub posted_content {
  return %{$lc_session->{'content'}};
}

sub get_posted_content {
  my ($r)=@_;
  my $query = new CGI($r);
  my %content=$query->Vars;
  ...
  $lc_session->{'content'}=\%content;
}
\end{lstlisting}

{\tt get\_session()} in {\tt lc\_auth\_acc.pm} calls {\tt get\_posted\_content()} in {\tt lc\_entity\_sessions.pm} which uses CGI to load the query string into the global hash variable, {\tt \$lc\_session}.  {\tt lc\_ui\_portfolio.pm}, called by {\tt lc.conf} does most of the work of populating the portfolio.

\begin{lstlisting}[language=Perl,frame=single,title=lc\_ui\_portfolio.pm]
sub handler
  my %content=&Apache::lc_entity_sessions::posted_content();
  ...
  if ($content{'command'} eq 'listdirectory') {
    # Do a directory listing
    $r->content_type('application/json; charset=utf-8');
    $r->print(&listdirectory($path,$content{'showhidden'}));

sub listdirectory
  ...
  my $output;
  $output->{'aaData'}=[];
  ...
  my $dir_list=&Apache::lc_entity_urls::full_dir_list($path);
  foreach my $file (@{$dir_list}) {
    ...
    push(@{$output->{'aaData'}},
    ...
  }
  return &Apache::lc_json_utils::perl_to_json($output);
\end{lstlisting}

This handler calls {\tt posted\_content()} from {\tt lc\_entity\_sessions.pm} to retrieve the contents of {\tt \$lc\_session}.  Recall the {\tt command=listdirectory} part of the {\tt sAjaxSource} query. {\tt lc\_ui\_portfolio.pm} checks for this and then proceeds to call {\tt listdirectory()}.  

{\tt listdirectory()} creates a hash reference called {\tt \$output} to hold the contents of the portfolio.  The {\tt 'aaData'} entry is defined as an empty list.  The metadata for each portfolio element is pushed to this list.  Finally, {\tt perl\_to\_json()} is used to convert the list containing hash reference, {\tt \$output}, to a string in proper JSON format.  The perl module {\tt JSON::DWIW} (``Does what I want'') handles the conversion.

The JSON object is returned as the arguement of {\tt \$('\#portfoliolist').dataTable()} in {\tt lc\_ui\_portfolio.js} which originally made the request.  The portfolio table is populated and displayed.

\subsubsection{Upload content}
From the portfolio page, new content can be added by clicking the ``Upload file'' button.  Generating the upload button involves the XML parsing described in Chapter~\ref{xmlparser}.  {\tt lc\_portfolio.html} contains the line:

\begin{lstlisting}[language=HTML,frame=single,title=lc\_portfolio.html]
<lcfileupload id="newfile" target='/upload_file' successcall="uploadsuccess" 
    failurecall="uploadfailure" /> 
\end{lstlisting}

{\tt lcfileupload} is a custom tag that must be interpreted.  {\tt lc.conf} parses HTML files with {\tt lc\_asset\_xml.pm}.

\begin{lstlisting}[language=apache,frame=single,title=lc.conf]
<LocationMatch "(?i)\.(xml|html|htm|xhtml|xhtm|problem)$">
SetHandler perl-script
PerlHandler Apache::lc_asset_xml
</LocationMatch> 
\end{lstlisting}

\newpage

\begin{lstlisting}[language=Perl,frame=single,title=lc\_asset\_xml.pm]
sub handler {
  my $r = shift;
  my $fn=$r->filename();
  ...
    $r->print((&target_render($fn,['html'],{}))[0]);
  ...
}

sub target_render {
  my ($fn,$targets,$stack,$content,$context,$outputid)=@_;
  ...
  my $p=HTML::TokeParser->new($fn);
  ...
  my $output=&parser($p,$safe,$stack,$status,$targets->[-1]);
  return ($output,$stack);
}
\end{lstlisting}

The main handler passes the filename to {\tt \&target\_render()} along with the target, 'html'.  {\tt \&target\_render()} creates a new parser object for {\tt lc\_portfolio.html} and passes it to {\tt \&parser()}.

\begin{lstlisting}[language=Perl,frame=single,title=lc\_asset\_xml.pm (cont.)]
sub parser {
  my ($p,$safe,$stack,$status,$target)=@_;
  ...
  while (my $token = $p->get_token) {
    ...
    } elsif ($token->[0] eq 'S') {
      # A start tag - evaluate the attributes in here
      ...
      $tmpout=&process_tag('start',$token->[1],$p,$safe,$stack,$status,$target,$token);
      ...
      $output.=$tmpout
  ...
  return $output;
}

sub process_tag {
  my ($type,$tag,$p,$safe,$stack,$status,$target,$token)=@_;
  ...
  my $outtag=$type.'_'.$tag.'_'.$target;
  ...
  }
  ...
  if (defined(&$outtag)) {
    $tag_output.=&{$outtag}($p,$safe,$stack,$token);
  ...
}
\end{lstlisting}

The {\tt while} loop processes each line. The {\tt <lcfileupload/>} statement is considered a start tag and is passed to {\tt \&process\_tag()}.  {\tt \&process\_tag()} calls {\tt \&start\_lcfileupload\_html()} imported from {\tt lc\_xml\_forms.pm}.

\begin{lstlisting}[showstringspaces=false, language=Perl,frame=single,title=lc\_xml\_forms.pm]
sub start_lcfileupload_html {
  my ($p,$safe,$stack,$token)=@_;
  my $id=$token->[2]->{'id'};
  my $name=$token->[2]->{'name'};
  my $target=$token->[2]->{'target'};
  my $description=$token->[2]->{'description'};
  my $success=$token->[2]->{'successcall'};
  my $fail=$token->[2]->{'failurecall'};
  unless ($name) { $name=$id; }
  unless ($description) { $description="Upload file"; }
  my $output='<label class="lcfileuploadlabel" for="'.$id.'" id="'.$id.'label">'
      .&mt($description).'</label>';
  $output.='<input id="'.$id.'" name="'.$name.'" class="lcinnerfileupload"
      type="file" onChange="do_upload(this.form,event,'."'$target','$id',
      '$success','$fail'".')" />';
  $output.=&hidden_field($id.'_path','');
  return $output;
}
\end{lstlisting}

The following HTML is generated and ultimately passed to the client:

\begin{lstlisting}[showstringspaces=false, language=HTML,frame=single]
<label id="newfilelabel" class="lcfileuploadlabel" for="newfile">Upload file</label>
<input id="newfile" class="lcinnerfileupload" type="file" 
    onchange="do_upload(this.form,event,'/upload_file','newfile',
    'uploadsuccess','uploadfailure')" name="newfile">
</input> 
\end{lstlisting}



%{\tt load\_path()} which accesses another Perl handler with the request {\tt \$.getJSON( "/portfolio", ...)}.  {\tt lc.conf} directs "portfolio" requests to {\tt loncapa/app/handlers/lc\_ui\_portfolio.pm} which returns the contents of the user's portfolio.

Many of the menu options follow this pattern.  Menu options execute JavaScript functions which replace the contents of the iframe through a call to {\tt display\_asset()}.  The important contents of the page can be quickly changed without having to rebuild the entire page.  The replacement HTML can load a JavaScript file containing any additional functions needed in that setting.  Those JavaScript files can communicate with the server through JSON requests. {\tt lc.conf} directs the JSON request to  a specific Perl handler depending on the given path.

As one more example, let's see what happens when we go to Preferences.  Clicking Preferences from the menu calls the JavaScript function {\tt preferences()} in {\tt lc\_defaults.js}.  This in turn calls {\tt display\_asset("/pages/lc\_course\_preferences.html")} to replace the iframe.  {\tt lc\_course\_preferences.html} loads {\tt loncapa/app/scripts/lc\_preferences.js}  The only server communication occurs when the user clicks the "Store" button.  Since data is being passed {\it to} the server opposed to just being retrieved, an AJAX POST request is used instead of the GET request ({\tt \$.getJSON}).  The options set in the preferences form are passed with 

{\tt \$.ajax(\{url: "/preferences", type: "POST", ...\})} \newline {\tt lc.conf} directs "preferences" calls to {\tt loncapa/app/handlers/lc\_ui\_preferences.pm}.


\subsection{Where should I be working?}
If you are looking to change the structure or appearance of a page, then you will probably want to edit its associated HTML file located in {\tt loncapa/app/html/pages}.  Adding or changing client-side functionality will require editing the page's JavaScript file located in {\tt loncapa/app/scripts}.

Server-side functionality is again controlled by Perl handlers located in {\tt loncapa/app/handlers}.  Data is passed to and retrieved from the server through AJAX/JSON requests that are placed in the page's JavaScript file.  Perl handlers operate at a low level and should not need to be modified very often.  There is probably already a Perl handler that will meet your needs.