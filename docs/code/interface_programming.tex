\chapter{Interface Programming}
\label{interfaceprogramming}
\section{Interface Pages}


\subsection{General Mechanism}
Interface pages (as opposed to asset pages) in LON-CAPA are typically controlled by an HTML, a JavaScript, and a Perl script file located in the {\tt loncapa/app} directory.
\begin{itemize}
\item  {\tt loncapa/app/html/pages} and {\tt loncapa/app/html/modals} contain the HTML/XML framework which is displayed to the user. The page contains the general layout, and it combines HTML-tags with LON-CAPA XML-tags. As this
page is processed by the parser (see Chapter~\ref{xmlparser} on page~\pageref{xmlparser}), the LON-CAPA tags are replaced by dynamically generated HTML on the way to the browser.
\item The HTML file will usually load JavaScript files located in {\tt loncapa/app/scripts} to control client-side functionality.
\item As the scripts make AJAX requests\index{AJAX} to update parts of the page, server-side functionality is provided by Perl modules located in {\tt loncapa/app/handlers}. These modules often return simple text or JSON,\index{JSON} which is processed client-side and injected into the page.
\end{itemize}
In rare cases, XML-tags establish areas which are not updated by client-side JavaScript, but which require server-side functionality. Usually, this is the case when server-provided data is not
structured enough to map well into JSON, e.g., when targeted follow-up questions in various formats need to be generated on the fly. The handlers for this server-side generated HTML, which is then
directly injected into the page, are located in {\tt loncapa/xml/xml\_includes}.


\subsection{Examples}
Let's walk through a couple examples to see how these different files are used.  We begin at the homepage of the site which loads {\tt loncapa/app/html/index.html}.  This sets the general structure of the page and defines space for things like the page header and the menu.  It also loads the JavaScript file, {\tt loncapa/app/scripts/lc\_default.js}, which includes common JavaScript functions.  One such function that is called upon loading is {\tt menubar()}.  This function dynamically generates the menu based on the context.  It communicates with the server through a JSON call, {\tt \$.getJSON("menu",...)}.  {\tt  lc.conf} directs all "menu" requests to the Perl handler, {\tt loncapa/app/scripts/lc\_ui\_menu.pm}.  The handler returns a JSON object containing the appropriate menu data.

Another function in {\tt lc\_defaults.js} is {\tt display\_asset(newuri)} which replaces the main iframe with the contents of the passed uri.  {\tt lc\_default.js} also calls the JavaScript function, {\tt dashboard()}, which loads the Dashboard.  It uses {\tt display\_asset()} to load {\tt lc\_dashboard.html} into the iframe.

Now let's see what happens when we go to the Portfolio space.  Clicking on the menu option, Portfolio, calls the JavaScript function, portfolio(), which uses {\tt display\_asset()} to load {\tt lc\_portfolio.html} into the iframe.  {\tt lc\_portfolio.html} loads {\tt lc\_portfolio.js} containing additional JavaScript functions needed in the portfolio space.  Upon loading, {\tt lc\_portfolio.js} calls {\tt load\_path()} which accesses another Perl handler with the request {\tt \$.getJSON( "/portfolio", ...)}.  {\tt lc.conf} directs "portfolio" requests to {\tt loncapa/app/handlers/lc\_ui\_portfolio.pm} which returns the contents of the user's portfolio.

Many of the menu options follow this pattern.  Menu options execute JavaScript functions which replace the contents of the iframe through a call to {\tt display\_asset()}.  The important contents of the page can be quickly changed without having to rebuild the entire page.  The replacement HTML can load a JavaScript file containing any additional functions needed in that setting.  Those JavaScript files can communicate with the server through JSON requests. {\tt lc.conf} directs the JSON request to  a specific Perl handler depending on the given path.

As one more example, let's see what happens when we go to Preferences.  Clicking Preferences from the menu calls the JavaScript function {\tt preferences()} in {\tt lc\_defaults.js}.  This in turn calls {\tt display\_asset("/pages/lc\_course\_preferences.html")} to replace the iframe.  {\tt lc\_course\_preferences.html} loads {\tt loncapa/app/scripts/lc\_preferences.js}  The only server communication occurs when the user clicks the "Store" button.  Since data is being passed {\it to} the server opposed to just being retrieved, an AJAX POST request is used instead of the GET request ({\tt \$.getJSON}).  The options set in the preferences form are passed with 

{\tt \$.ajax(\{url: "/preferences", type: "POST", ...\})} \newline {\tt lc.conf} directs "preferences" calls to {\tt loncapa/app/handlers/lc\_ui\_preferences.pm}.


\subsection{Where should I be working?}
If you are looking to change the structure or appearance of a page, then you will probably want to edit its associated HTML file located in {\tt loncapa/app/html/pages}.  Adding or changing client-side functionality will require editing the page's JavaScript file located in {\tt loncapa/app/scripts}.

Server-side functionality is again controlled by Perl handlers located in {\tt loncapa/app/handlers}.  Data is passed to and retrieved from the server through AJAX/JSON requests that are placed in the page's JavaScript file.  Perl handlers operate at a low level and should not need to be modified very often.  There is probably already a Perl handler that will meet your needs.

%Many of the menu options behave this way by replacing the iframe with simple HTML files.  The important contents of the page can be quickly changed without having to rebuild the entire page.  Most of the necessary JavaScript functions are already available after the initial loading of {\tt lc\_defaults.js}.

%One of the menu options is Portfolio which onClick calls the JavaScript function, portfolio()

%{\tt loncapa/app/scripts/lc\_default.js} also calls the JavaScript function, {\tt dashboard()}, which loads the Dashboard.  The function {\tt  diaplay\_asset(“/pages/lc\_dashboard.html”)} replaces the main iframe with the contents of {\tt lc\_dashboard.html}.

%Let's look at the Portfolio space as an example to see how these different files are used.  Clicking on {\tt Places->Portfolio} from the menu runs a JavaScript function which loads {\tt loncapa/app/html/pages/lc\_portfolio.html}.  This is a minimal page that defines the framework of HTML elements and loads the JavaScript file {\tt loncapa/app/scripts/lc\_portfolio.js}, in addition to other scripts that are loaded when the parser processes the {\tt $<$html$>$}-tag.
