\chapter{Glossary}
\begin{description}
\item[Asset] An HTML page, homework problem, image, movie, etc.
\item[Author] A user who has author privileges, i.e., who can publish assets
\item[Cluster] Group of machines linked into a network governed by a cluster table
\item[Course] An entity with members and a table of contents. Used as a general term to also include communities
\item[Community] Special kind of ``course'' without a gradebook, usually used for collaboration
\item[Domain] Logical segmentation of a LON-CAPA cluster, usually by institution
\item[Entity Code] Part of the unique identifier of assets, users, and courses, the other part being the domain
\item[Group] A special kind of section, mostly for teamwork. Students can be in more than one section at a time
\item[Homeserver] Node in a cluster which holds the authoritative and permanent copy of a user's or course's data and assets
\item[Portfolio] The space where users store their assets
\item[Realm] The extend of privileges, i.e., system-wide, domain-wide, course-wide, section-wide, or user-wide
\item[Role] Users can have a number of (time-limited) roles which each grant a certain set of privileges within a particular realm
\item[Section] Sub-group of a course, for example a lab section. A student can only be in one section at a time
\item[WYSIWYG] What You See Is What You Get, meant to look the same when edited and rendered (on the web in LON-CAPA's context)
\item[WYSIWYM] What You See Is What You Mean, displayed in the best way to convey the semantics
\end{description} 
