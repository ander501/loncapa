\chapter{Internationalization}\index{internationalization}
\section{Encoding}\index{encoding}
Unicode\index{unicode} UTF-8\index{utf-8} is used throughout. All assets, program code, translation files, stored data, etc, is stored in utf-8 in order to avoid encoding problems down the road.
\section{Maketext}
LON-CAPA uses Maketext for internationalization, both inside of documents and called programatically. For each language, a translation file needs to be written. The format is simple:
\begin{verbatim}
package Apache::lc_localize::de;
use base qw(Apache::lc_localize);
use utf8;

%Lexicon=('_AUTO' => 1,
'language_code'      => 'de',
'language_direction' => 'ltr',
'language_description' => 'Deutsch',
'date_locale'  => '$weekday, $day. $month $year, $twentyfour:$minutes:$seconds Uhr',
'date_short_locale' => '$day.$month.$year',
'date_months'  => 'Jan.,Feb.,März,April,Mai,Juni,Juli,Aug.,Sep.,Okt.,Nov.,Dez.',
'date_days'    => 'So.,Mo.,Di.,Mi.,Do.,Fr.,Sa.',
'date_am' => 'vormittags',
'date_pm' => 'nachmittags',
'date_format' => '24',

'superuser' => 'Superuser',
'domain_coordinator' => 'Domänenkoordinator',
'course_coordinator' => 'Kurskoordinator',
'instructor' => 'Dozent',
'teaching_assistant' => 'Tutor',
'student' => 'Studierender',
'community_organizer' => 'Gemeinschaftsorganisator',
'member' => 'Mitglied',
'author' => 'Autor',
'co_author' => 'Co-Autor',

'Modify Selected Entries' => 'Ausgewählte Einträge ändern',

'Add New Entry' => 'Neuen Eintrag hinzufügen',

"Showing [_1] to [_2] of [_3] entries" => "Einträge [_1] bis [_2] von [_3] insgesamt",

...
);

1;
__END__
\end{verbatim}
The files need to be utf-8-encoded, and we need to make sure that Perl knows this by specifying ``use utf8.''

Some configuration entries such as date\_locale are mandatory. Where translations are missing, English is used. The [\_{\it n}] are placeholders for variables.
\section{Invocation in documents}
In documents, phrases can be translated using the $<$localize$>$-tag, e.g.
\begin{verbatim}
<span class="lcsuccess"><localize>Your preferences were saved.</localize></span>
\end{verbatim}
Placeholders can be defined using the parameter argument, e.g., $<$localize parameters="42,'Fred'"$>$.
This tag is defined in module lc\_xml\_localize.
\section{Invocation from Perl}
Internationalization is available from Perl via the \%mt()-function. The first argument is the text, any following arguments come from the remaining arguments, e.g., 
\begin{verbatim}
&mt("Showing [_1] to [_2] of [_3] entries",$start,$finish,$total)
\end{verbatim}
\section{Internal codes}
Translation should happen as close to the interface level as possible. If server-side functionality needs to store or generate things that need to be used both by humans and by other application components, those should not be translated too early. A good way are internal codes, and an example are roles.\index{roles} They are stored by codes and translated on the way out.

The file lc\_localize.pm has such Code to English ``translations,'' e.g.,
\begin{verbatim}
package Apache::lc_localize::en;
use base qw(Apache::lc_localize);
%Lexicon=('_AUTO' => 1, 
'language_code'      => 'en',
'language_direction' => 'ltr',
'language_description' => 'English',
'date_short_locale' => '$month/$day/$year',

'superuser' => 'Superuser',
'domain_coordinator' => 'Domain Coordinator',
'course_coordinator' => 'Course Coordinator',
'instructor' => 'Instructor',
'teaching_assistant' => 'Teaching Assistant',
'student' => 'Student',
'community_organizer' => 'Community Organizer',
'member' => 'Member',
'author' => 'Author',
'co_author' => 'Co-Author'
);
\end{verbatim}
\section{Special cases}
\subsection{JavaScript configuration files}
Some Javascript utilities expect a configuration file. This can be generated dynamically like for example in lc\_ui\_datatable\_i14n, which is invoked by lc.conf:
\begin{verbatim}
sub handler {
# Get request object
   my $r = shift;
   $r->content_type('application/json; charset=utf-8');
   my $items={
    "sEmptyTable"    => &mt("No data available in table"),
    "sInfo"          => &mt("Showing [_1] to [_2] of [_3] entries",'_START_','_END_','_TOTAL_'),
    "sInfoEmpty"     => &mt("Showing 0 to 0 of 0 entries"),
    "sInfoFiltered"  => &mt("(filtered from [_1] total entries)",'_MAX_'),
    "sInfoPostFix"   => "",
    "sInfoThousands" => &mt(","),
    "sLengthMenu"    => &mt("Show [_1] entries",'_MENU_'),
    "sLoadingRecords"=> &mt("Loading..."),
    "sProcessing"    => &mt("Processing..."),
    "sSearch"        => &mt("Search:"),
    "sZeroRecords"   => &mt("No matching records found"),
    "oPaginate" => {
        "sFirst"    => &mt("First"),
        "sLast"     => &mt("Last"),
        "sNext"     => &mt("Next"),
        "sPrevious" => &mt("Previous")
                   },
    "oAria" => {
        "sSortAscending"  => &mt(": activate to sort column ascending"),
        "sSortDescending" => &mt(": activate to sort column descending")
               }
   };
   $r->print(&Apache::lc_json_utils::perl_to_json($items));
   return OK;
}
\end{verbatim}
This corresponds to the code on the page,
\begin{verbatim}
$('#courselist').dataTable( {
      "bStateSave": true,
      "oLanguage" : {
         "sUrl" : "/datatable_i14n"
      },
      "aoColumns" : [
         { "bSortable": false },
         null,
         null,
         {"iDataSort": 4},
         {"bVisible": false}
      ]
    } );
\end{verbatim}
which loads it.
\subsection{Loading locatilization files}
Some client-side JavaScript has its own localization files as part of the distibution. Here, the LON-CAPA localization file can be used to pick the right one. For example, the CKeditor gets its language from the HTML tag
\begin{verbatim}
<html lang="de" dir="ltr">
\end{verbatim}
which is based on the German translation file
\begin{verbatim}
'language_code'      => 'de',
'language_direction' => 'ltr',
\end{verbatim}
--- but it could also take a separate language parameter, if ever these languages need to differ.

