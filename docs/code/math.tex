\chapter{Math editor, parser and equation syntax}\index{math}

\section{Introduction}

Historically, LON-CAPA has used several syntaxes for equations, such as Maxima and R.
These syntaxes have evolved on the client side to be easier to write, in particular with implicit operators.
Yet, many students have struggled to enter equations correctly. The consequences for incorrectly
writing an equation vary depending on the context, from a simple admonition to a loss of points in an exam.

It is essentially a user interface problem, but the interpretation of a complex syntax can require
a lot of code, and with implicit operators a correct interpretation actually depends on the context.
For evaluations, LON-CAPA 2 uses some Perl transformations, Maxima, R,
and some old C/lex\&yacc code based on CAPA.
It has become difficult to make this system evolve or even to maintain it.

To improve the user interface, limit the risk of misinterpretation and allow for future customization,
the following decisions were made:
\begin{itemize}
\item The user interface needs a real-time feedback, so that users never send an equation with a syntax error.
It should also help them to learn the syntax.
\item Users should use a single, well-defined syntax
(other syntaxes might be kept running for backward-compatibility).
\item New server-side code should be written to replace everything that does not require a CAS.
It should be flexible to allow for changes in the syntax, and easier to maintain.
\item On the client and server sides, the equation parser has to understand mathematical expressions with units,
variables and constants.
\end{itemize}


\section{A new user interface}\index{equation editor}

LON-CAPA 2 provides 2 user interfaces to write equations: one with a simple text field
(users need to know the syntax), and one with a WYSIWYG Java applet. Due to new system
and browser incompatibilities with Java, the applet does not work anymore in many environments.

A WYSIWYG editor is a popular demand today, but we have not found a libre, easy to use and reliable
implementation in Javascript. Also, it might not be the best interface for end users: while it is
nice to be able to input equations without having to know the syntax, using a WYSIWYG editor
is not very easy for users who will have to enter many equations, and it hides some mathematical information.
Writing a new WYSIWYG editor from scratch would be too long given our current developer team.
This situation could change in the future, as a libre Javascript WYSIWYG editor will probably be
created eventually.

An alternative today is to use a real-time preview for the equation, showing how the equation is interpreted
as it is typed. The preview also needs to show the position of a syntax error.
The text field + real-time preview solution helps to avoid sending an equation with a bad syntax, or
an equation that would be misinterpreted. It also helps to learn operator precedence,
as \texttt{1+2/3} will not show up like \texttt{(1+2)/3}.

This solution was chosen for the new user interface. The implementation uses MathJax to display
equations, reducing the Javascript code to a parser exporting equations to MathML.
As with LON-CAPA 2, equations may use implicit operators. Since it is not always possible to
guess automatically if a name is representing a variable or a unit, two separate modes
are used to interpret equations: \textit{unit}, for expressions with units and constants,
and \textit{symbolic}, for expressions with variables and constants.
A short list of constants is used to recognize them.

The editor can be tried from the git by opening the following file in a web browser
(it does not require a LON-CAPA install):
\begin{verbatim}
loncapa/app/scripts/maxima_editor/test.html
\end{verbatim}

\subsection{Usage}
In an HTML, script elements are used to reference the editor minimized version and MathJax.
A new text area with automatic equation preview can then be inserted in the document with the math class:
\begin{verbatim}
<textarea class="math" data-implicit_operators="true" data-unit_mode="true"
    data-constants="c, pi, e, hbar, amu, G" spellcheck="false" autofocus="autofocus"></textarea>
\end{verbatim}


\section{The server-side parser}\index{equation parser}

The server-side parser is similar to the one written in Javascript, and in fact is a direct translation of it
in Perl to make it easier to maintain.
The MathML export is replaced by a calculation code and CAS exports. It also uses the unit and
symbolic modes.
The parser is located in the git at \texttt{math/math\_parser}.
Two test scripts can be used to try it, make sure it works well, and learn usage:
\texttt{test/math\_parser\_test\_cases.pl} and \texttt{test/math\_parser\_manual\_test.pl}
(they require a LON-CAPA install to work).

\subsection{Usage}
The parser constructor takes 2 parameters: \texttt{implicit\_operators}, a boolean (1 or 0) that
tells if implicit operators are allowed, and
\texttt{unit\_mode}, another boolean to specify the mode (unit or symbolic).
The parser's parse takes an equation string as a parameter, and returns an \texttt{ENode} representing
a parsed tree for the equation. This \texttt{ENode} object can then be used for conversion or calculation.

\subsection{Calculation environment}
An object of the class \texttt{CalcEnv} is used to specify the mode (unit or symbolic), additional units,
and variable values. It is passed to the \texttt{calc} method of \texttt{ENode} for calculation.

\subsection{Conversion to Maxima syntax}
An \texttt{ENode} can be used for a Maxima syntax export, with the \texttt{toMaxima()} method.

\subsection{Catching errors}
Exceptions are thrown whenever a problem occurs.
\texttt{ParseException} and \texttt{CalcException} can be caught with try/catch,
and it is possible to get the raw error message or a localized one.

\subsection{Example with unit node}
\begin{verbatim}
    my $implicit_operators = 1;
    my $unit_mode = 1;
    my $equation = "4 peck + 2 bushel";
    my $p = Parser->new($implicit_operators, $unit_mode);
    my $root = $p->parse($equation);
    my $env = CalcEnv->new($unit_mode);
    $env->setUnit("peck", "2 gallon");
    $env->setUnit("bushel", "8 gallon");
    $env->setUnit("gallon", "4.4 L");
    print "Maxima syntax: ".$root->toMaxima()."\n";
    print "Value: ".$root->calc($env)->toString()."\n";
\end{verbatim}

\subsection{Example with symbolic node}
\begin{verbatim}
    my $implicit_operators = 1;
    my $unit_mode = 0;
    my $equation = "1/(x-y-z)";
    my $p = Parser->new($implicit_operators, $unit_mode);
    my $root = $p->parse($equation);
    my $env = CalcEnv->new($unit_mode);
    $env->setVariable("x", 1/2);
    $env->setVariable("y", 1/3);
    $env->setVariable("z", 1/7);
    print "Maxima syntax: ".$root->toMaxima()."\n";
    print "Value: ".$root->calc($env)->toString()."\n";
\end{verbatim}


\section{Equation syntax}\index{equation syntax}

\subsection{Spaces}
Spaces are always ignored.

\subsection{Decimal separators and function parameter separators}
By default, \texttt{","} and \texttt{"."} can be used as decimal separators.
\texttt{";"} is used to separate function and vector/matrix parameters.

Variables
Variable names are used directly, without any special character before.

\subsection{Constants}
Constant names are used directly. LON-CAPA has a list of known constants.

\subsection{Units}
Unit names are used directly. LON-CAPA has a list of known units.

\subsection{Parenthesis}
Parenthesis can be used to specify evaluation order.

\subsection{Complex numbers}
Complex numbers are understood, and will be used in calculations.
This means that \texttt{"i"} should never be used as a variable.

\subsection{Operators}
\begin{itemize}
\item arithmetic: \texttt{+ - * / \^}
\item factorial: \texttt{!}
\item relational: \texttt{= \# < <= >= >}
\item percent of a constant: \texttt{\%}
Example: \texttt{2\%c = 2/100*c}
\item units: \texttt{`}
Example: \texttt{2`m + 3`m = 5`m}
\end{itemize}

\subsection{Implicit operators}
\texttt{*} and \texttt{`} are implicit in LON-CAPA.
The parser will try to guess which operator is missing whenever possible.
The choice between \texttt{*} and \texttt{`} depends on the mode for interpreting equations.
Example: \texttt{2c+3m/s} is understood in unit mode to be \texttt{2*c + 3`(m/s)}.
In symbolic mode, it would be interpreted \texttt{2*c + (3*m)/s} (\texttt{m} and \texttt{s} being variables).

\subsection{Vectors and matrices}
Vectors and matrices are defined with square brackets. A matrix is made of a list of row vectors.
\begin{itemize}
\item vectors: \texttt{[1;2;3]}
\item matrices: \texttt{[[1;2];[3,4;5,6]]}
(remember the comma is a decimal separator)
\end{itemize}

\subsection{Functions}
Functions use the syntax \texttt{f(a;b)}.
\begin{itemize}
\item basic:
\begin{verbatim}
    pow(x;y)=x^y, sqrt(x), abs(x), exp(x)=e^x, factorial(x)=x!,
    ln(x), log(x)=ln(x), log10(x)=ln(x)/ln(10)
\end{verbatim}
\item less common functions:
\begin{verbatim}
    mod(x;y) (modulo), sgn(x) (sign of x), ceil(x) (ceiling), floor(x),
    binomial(n;p)=n!/(p!*(n-p)!);
\end{verbatim}
\item requiring symbolic mode:
\begin{verbatim}
    sum(f(x);x;x1;x2), product(f(x);x;x1;x2)
\end{verbatim}
"i" cannot be used as a variable name because it can be confused with the imaginary number.
Example: \texttt{sum(a\^2; a; 1; 5)}
\item requiring a CAS like Maxima for calculation:
\begin{verbatim}
    diff(expr; x; n), integrate(expr; x; a; b), limit(expr; x; val; dir)
\end{verbatim}
The syntax is the same as Maxima.
\item trigonometry:
\begin{verbatim}
sin(x), cos(x), tan(x), asin(x), acos(x), atan(x), atan2(x;y),
sinh(x), cosh(x), tanh(x), asinh(x), acosh(x), atanh(x)
\end{verbatim}
\end{itemize}
